\section{Wybór metryk}
Metryki statyczne mają swoje zastosowanie w mierzeniu parametrów kodu i ich analiza może przyczynić się do znacznej poprawy jego jakości, przekładając się tym
na większą produktywność i w konsekwencji wartość biznesową. Ewentualna produktywność wynikająca z ich zastosowania nie jest jednak mierzalna za ich pomocą, a właśnie
na to kładzie nacisk ta praca. Z tego powodu w jej dalszej części nie są one już rozpatrywane.

Pomimo niezaprzeczalnej dominacji Scruma pod kątem popularności i stopnia jego zaadaptowania w ogromnej liczbie organizacji, nie definiuje on metryk pozwalających
na rzetelną analizę, a do weryfikacji rezultatów i pomiaru produktywności musi posiłkować się innymi metodami, wahającymi się z organizacji na organizację.
W kontraście, Kanban definiuje metryki niezależne od siebie nawzajem, a więc do pewnego stopnia odporne na świadomą manipulację nimi, pozwalając na bardziej rzetelną analizę.
Wiek pracy w toku oraz przepustowość pomimo niedokładności, pozwalają jednak na szybką reakcję w organizacji projektu, co czyni tę metodykę prawdziwie zwinną.
Efektywność przepływu pozwala na identyfikację powodu stojącego za wydajnością zespołu, niezależnie czy jest ona wysoka czy niska.
Czas realizacji pozwala z kolei na weryfikację zobowiązań biznesowych, ułatwiając przy tym estymację przy kontakcie z klientem.
Z powodu bogatego wachlarza możliwości, do realizacji metryki Kanbanowe, zwłaszcza że można wykorzystywać je również w Scrumie, co zwiększa uniwersalność projektu.
