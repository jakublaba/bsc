\newpage
\section{Metryki Statyczne}
Pierwszym rodzajem omawianych metryk są metryki statyczne. Spośród omawianych w tej pracy metryk wyróżniają się tym,
że jako jedyne nie skupiają się na mierzeniu parametrów pracy zespołu, tylko kodu.
Część z nich jest w dzisiejszych czasach zintegrowana w narzędziach deweloperskich, takich jak \href{https://www.synopsys.com/software-integrity/static-analysis-tools-sast/coverity.html}{Coverity}
lub \href{https://www.sonarsource.com/products/sonarlint/}{SonarLint}.

\subsection*{Liczba linii kodu (LOC)}
Najprostszą metryką statyczną jest liczba linii kodu (ang. Lines of Code -- LOC). Metryka ta skupia się stricte na analizie kodu źródłowego, aby przełożyć
ją na metrykę mierzącą pracę zespołu, można mierzyć LOC na dany okres.
Jest to dobra metryka do orientacyjnej oceny rozmiaru projektu czy danej funkcjonalności, jednak nie nadaje
się do dokładniejszych analiz, ze względu na fundamentalne wady.

\subsubsection*{Mniej linii nie zawsze oznacza lepszy kod}
Jako przykład można przyjrzeć się tzw. \textit{fluent API}, czyli interfejsom programistycznym zaprojektowanym z myślą o łańcuchowym wywoływaniu metod.
Przykładem takich interfejsów są np. Stream API (Java), AssertJ (Java), JooQ (Java), LINQ (C\#).
Przy korzystaniu z tego typu interfejsów, wywoływanie kolejnych metod w pojedynczej linii jest bardzo nieczytelne, zalecane jest
umieszczanie każdego wywołania metody w nowej linii, co wpływa negatywnie na metrykę LOC.
\begin{lstlisting}[caption=Sumowanie liczb pierwszych przy użyciu Stream API (Java)]
var primesSum = IntStream.range(1, 1_000)
    .filter(SomeLibrary::isPrime)
    .sum();
\end{lstlisting}

\subsubsection*{Różnice składni i stylu kodu}
Niektóre języki mają po prostu bardziej zwięzłą składnię niż inne.
Przykładowo, JavaScript i Kotlin posiadają operator \texttt{?}, pozwalający pisać w 1 linii kod, który w Javie zająłby 4:
\begin{lstlisting}[caption=Przykład obsługi null (Kotlin)]
val obj = a?.field;
\end{lstlisting}
\begin{lstlisting}[caption=Przykład obsługi null (Java)]
Object obj = null;
if (a != null) {
    obj = a.field;
}
\end{lstlisting}

\subsubsection*{Refaktoryzacja}
Istotnym elementem utrzymywania dobrej jakości kodu w dużych projektach informatycznych jest refaktoryzacja, czyli niefunkcjonalne zmiany nie niosące stricte dodatkowej wartości biznesowej,
a jedynie polepszające jakość kodu w celu ułatwienia jego utrzymania go w przyszłości.
Częstym elementem refaktoryzacji jest usuwanie tzw. \textit{martwego kodu}, czyli kodu który nie jest wykorzystywany -- np. gałęzie wyrażeń \texttt{switch} czy \texttt{if} które są nieosiągalne, nieużywanych metod, zmiennych czy testów.
Innym elementem refaktoryzacji bardzo często jest skracanie istniejącego kodu do bardziej czytelnej formy.
W przypadku mierzenia LOC na okres do analizy pracy, w przypadku refaktoryzacji ta metryka może fałszywie wskazać ujemną wydajność zespołu.

\subsection*{Metryki Halsteada}
W 1977 prof Maurice Howard Halstead z Purdue University, sformułował metryki mające opisywać kod w dokładny sposób.\cite{halstead1977elements}
Metryki prof Halsteada opierają się o 4 wartości:
\begin{itemize}
    \item $\eta_1$ -- liczba unikalnych operatorów
    \item $\eta_2$ -- liczba unikalnych operandów
    \item $N_1$ -- całkowita liczba operatorów
    \item $N_2$ -- całkowita liczba operandów
\end{itemize}
Na ich podstawie można obliczyć następujące parametry:
\begin{itemize}
    \item Słownictwo: $\eta = \eta_1 + \eta_2$
    \item Długość: $N = N_1 + N_2$
    \item Szacunkowa długość: $\hat{N} = \eta_1log_2\eta_1 + \eta_2log_2\eta_2$
    \item Objętość: $V = N \times log_2\eta$
    \item Trudność: $D = \frac{\eta_1}{2} \times \frac{N_2}{\eta_2}$
    \item Wysiłek: $E = D \times V$
    \item Czas: $T = \frac{E}{18}[s]$
    \item Bugi: $B = \frac{E^{\frac{2}{3}}}{3000}$ Później uproszczone do $B = \frac{V}{3000}$
\end{itemize}
Metryki te nie przyjęły się, ze względu na ich właściwości:
\begin{enumerate}
    \item Objętość programu jest liniowo zależna od długości
    \item Wysiłek jest liniowo zależny od objętości
    \item Czas jest liniowo zależny od wysiłku
\end{enumerate}
Łańcuch tych zależności sprowadza się do tego, że każda z tych metryk jest u podstaw liniowo zależna od długości, która pod kątem przydatności niewiele różni się od liczby linii kodu. \\
Zebranie odpowiednich parametrów kodu i obliczenie tych metryk jest zatem inwestycją czasu, która w porównaniu z LOC jest nieopłacalna.
