\section{Metryki w metodyce Scrum}
Chociaż oficjalny Scrum Guide \cite{Schwaber2012TheSG} nie definiuje bezpośrednio metryk, powszechnie przyjęło się korzystanie z \textit{Velocity} oraz wykresów wypalenia.

\subsection{Czym są Story Points?}
Mimo że Scrum jest nieco starszy niż Extreme Programming, koncepcja Story Points została zaczerpnięta właśnie z XP.
Na swoim blogu \cite{StoryPointsRevisited} jeden z autorów tej metodyki opisuje że początkowo estymaty ustalało się po prostu w liczbie dni.
Nie była to jednak najbardziej precyzyjna jednostka do estymacji, a więc szybko został zaadaptowany koncept tzw. \textit{Dni Idealnych}.
Ron Jeffries opisuje dni idealne następującymi słowami: \textit{"We quickly went to what we called “Ideal Days”, which was informally described as how long it would take a pair to do it if the bastards would just leave you alone."}.
Kolokwialnie mówiąc, dni idealne to dni w których programiści mieliby święty spokój.
Aby z liczby dni idealnych uzyskać estymatę faktycznego czasu, były one mnożone przez współczynnik obciążenia.
Nazwa "dni idealne" często była skracana do "dni", co prowadziło do nieporozumień z klientami, stąd zaczęto nazywać je po prostu "punktami". \\
Obecnie Story Points funkcjonują jako relatywna jednostka, nie mająca bezpośredniego przełożenia na czas wykonania, a używana jedynie do oceny ilości pracy.

\subsection{Velocity}
Velocity jest bardzo prostą metryką, polegającą jedynie na zsumowaniu wagi zadań w danym sprincie, dając wgląd w wydajność zespołu.
Metryka ta niestety jest bardzo podatna na niepoprawne estymowanie zadań -- świadomie bądź nie, co może prowadzić do fałszywej oceny postępów w przypadku
zespołów które jeszcze się nie "skalibrowały" i niedoestymowują zadań, oraz zespołów które celowo je przeestymowują aby wypaść lepiej na papierze.

\subsection{Wykres wypalenia}
Błędnie nazywany metryką, ponieważ jest to jedynie wizualizacja postępu projektu.
Oś Y reprezentuje skumulowaną wagę pozostałych zadań, natomiast oś X czas.
Na wykresie wypalenia można zidentyfikować czy zespół nadąża z wykonywaniem zadań.
W dobrze przebiegającym projekcie powinien być obserwowalny trend spadkowy, moment kiedy wartość na osi Y spadnie do 0 jest momentem wykonania wszystkich zadań na zadany okres.

\subsection{Inne Metryki}
Jako że metryki nie są oficjalnie zdefiniowane, zależą one tak naprawdę od organizacji.
Firmy zajmujące się wspieraniem produkcji oprogramowania (np. \href{https://www.atlassian.com/}{Atlassian}, \href{https://www.sealights.io/}{Sealights}), wymieniają tutaj m.in. \cite{ScrumMetricsSealights} \cite{ScrumMetricsAtlassian}:
\begin{itemize}
    \item Sprint Goal Success
    \item Defect Density
    \item Time to Market
    \item ROI (Return of Investment)
\end{itemize}
Z powodu tego, że te metryki nie są mierzalne z samych parametrów zadań na platformie, nie są one istotne dla tej pracy.
