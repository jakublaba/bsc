\section{Metryki w metodyce Kanban}
Kanban jest metodyką zwinną służącą do zarządzania pracą intelektualną, gdzie jednym z podstawowych założeń jest zapewnienie mechanizmów
do wizualizacji tzw. \textit{niewidzialnej pracy}, czyli pracy intelektualnej.

Kanban wychodzi z założenia, że pojedyncza metryka nie jest wystarczająca --- każdy pojedynczy wyznacznik można łatwo oszukać. Z tego powodu, Kanban ma wiele metryk zasilających jego mechanizmy wizualizacji.

Projekt koncentruje się na kilku wybranych metrykach, opisanych w kolejnych podrozdziałach.

\subsection{Wiek pracy w toku}
Wiek pracy w toku (age of wip) odnosi się do poszczególnych zadań i opisuje od ilu dni trwa praca nad nimi.

Wiek Pracy w Toku pozwala szybko identyfikować blokujące się zadania poprzez odniesienie wartości metryki do danych projektu -- średniego czasu trwania zadań.

\subsection{Przepustowość}
Przepustowość (throughput) opisuje ile jednostek wartości dostarczamy w danym czasie. Metryka podobna do \textit{Velocity} znanego ze Scruma. Jednostką wartości mogą być \textit{storypoints}, zadania, funkcjonalności, itd.

Metryka ta daje bardzo uproszczony i ogólny wgląd w wydajność zespołu.

\subsection{Efektywność przepływu}
Efektywność przepływu (flow efficiency) jest nieco bardziej złożoną metryką: aby policzyć efektywność przepływu, rozpatrujemy ile czasu zadanie znajdowało się w stanie aktywnym (faktycznie była przeprowadzana nad nim praca), a ile czasu
w stanie oczekiwania (przykładowo statusy: \textit{``Selected for development''}, \textit{``Ready for testing''}, itd.). Wartość tej metryki to stosunek czasu spędzonego w stanie aktywnym do czasu w stanie oczekiwania.

W przypadku niskiej wydajności zespołu, efektywność przepływu pozwala zidentyfikować czy jest to spowodowane tempem pracy, czy nieoptymalnymi procedurami przekazywania zadań pomiędzy etapami.

\subsection{Czas realizacji}
Aby opisać czas realizacji (lead time), wpierw konieczne jest zdefiniowanie takich pojęć jak \textit{punkt zobowiązania} oraz \textit{punkt dostarczenia}.

Kanban jest zorientowany biznesowo, stąd punkt zobowiązania nie jest momentem faktycznego rozpoczęcia pracy nad zadaniem, tylko momentem zadeklarowania się przez zespół, że zadanie zostanie wykonane.
Przykładowo, w Scrumie punktem zobowiązania byłby sprint planning, a nie moment przeciągnięcia zadania ze statusu \textit{``TODO''} do \textit{``In Progress''}.
Punkt dostarczenia, to punkt oddania skończonego zadania --- w zależności od specyfiki zespołu może to oznaczać różne rzeczy, np.\ jeżeli zespół kontroluje cały produkt, to punktem dostarczenia
będzie faktyczne oddanie funkcjonalności klientowi, natomiast jeżeli mamy do czynienia z zespołem zajmującym się jedynie programowaniem, to punktem dostarczenia będzie przekazanie funkcjonalności do testerów.
Czas realizacji to czas jaki zadanie spędziło pomiędzy tymi dwoma punktami --- od zobowiązania do dostarczenia.

W przeciwieństwie do poprzednich metryk, ta jest bardziej przydatna biznesowi niż zespołowi wykonującemu projekt.
