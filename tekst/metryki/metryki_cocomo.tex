\section{Metryki w modelu COCOMO/COCOMO II}
COCOMO (Constructive Cost Model) jest modelem tworzenia kosztorysów dla projektów informatycznych, zaprojektowanym przez Barry'ego Boehma w 1981 roku. \cite[]{CocomoBoehm} COCOMO skupia się na szacowaniu kosztu oraz czasu projektu na podstawie metryk związanych z kodem oraz pracą zespołu.
W 1997 roku zaproponowano zaktualizowaną wersję, znaną jako COCOMO II, która została dostosowana do rozwoju trendów i technologii.

Najistotniejszą metryką w modelu COCOMO/COCOMO II, na której opiera się większość obliczeń jest liczba linii kodu źródłowego, wspomniana w rozdziale o metrykach statycznych. \ref{loc}

\subsection{Pracochłonność}
\label{cocomo:effort}
Pracochłonność (effort) określa ilość pracy potrzebną do ukończenia projektu. W modelu COCOMO II stosuje się wzór:
$E = a \times (SLOC^b) \times \prod_{i=1}^{n}E_i$,
gdzie:
\begin{itemize}
    \item $a$ - współczynnik zależny od typu projektu
    \item $b$ - współczynnik skalowania
    \item $SLOC$ - liczba linii kodu źródłowego
    \item $E_i$ - współczynniki kosztowe
\end{itemize}
Współczynniki kosztowe uwzględniają różne aspekty projektu, jak np. poziom doświadczenia zespołu, złożoność oprogramowania lub technologie stosowane w projekcie.

\subsection{Czas realizacji}
Czas realizacji (development time) określa czas potrzebny do ukończenia projektu. Jest obliczany na podstawie pracochłonności oraz innych czynników wpływających na tempo pracy. Stosuje się wzór:
$T = c \times E^d$,
gdzie:
\begin{itemize}
    \item $E$ - pracochłonność
    \item $c$, $d$ - współczynniki zależne od typu projektu
\end{itemize}
\newpage
\subsection{Wskaźniki skalowania}
\label{cocomo:scalefactors}
Wskaźniki skalowania (scale factors) mają istotny wpływ na obliczenia w modelu COCOMO II. Wyróżnia się pięć głównych:
\begin{itemize}
    \item PREC (Precedentness) - jak dobrze zdefiniowany jest projekt i czy zespół wcześniej miał do czynienia z czymś podobnym
    \item FLEX (Development flexibility) - poziom elastyczności w podejściu do rozwoju projektu
    \item RESL (Architecture / Risk resolution) - poziom analizy architektury i ryzyka
    \item TEAM (Team cohesion) - jak dobrze zespół współpracuje
    \item PMAT (Process maturity) - dojrzałość procesu, jak stabilny i przetestowany jest
\end{itemize}
Wskaźniki te są używane do obliczenia współczynnika $b$ w obliczaniu pracochłonności. \ref{cocomo:effort} Każdy wskaźnik jest ocenianiany na pięciostopniowej skali od "bardzo niski" do "bardzo wysoki",
wpływając na ostateczną wartość $b$.

\subsection{Współczynniki kosztowe}
Współczynniki kosztowe (cost drivers) to zestaw czynników wpływających na złożoność projektu. Model COCOMO II definiuje wiele takich współczynników, podzielonych na kategorie techniczne, organizacyjne i zespołowe. Niektóre z tych współczynników to:
\begin{itemize}
    \item RELY (Required software reliability) - wymagany poziom niezawodności oprogramowania
    \item DATA (Database size) - wielkość bazy danych w stosunku do kodu
    \item CPLX (Product complexity) - złożoność produktu
    \item TIME (Execution time constraint) - ograniczenie czasowe wykonania
    \item STOR (Main storage constraint) - ograniczenie dotyczące pamięci
    \item VIRT (Virtual machine volatility) - stabilność maszyny wirtualnej
\end{itemize}
Podobnie jak w przypadku współczynników skalowania \ref{cocomo:scalefactors}, każdy ze współczynników ma określony poziom wpływu, zastosowany we wzorze na pracochłonność \ref{cocomo:effort}, co pozwala
na dokładniejsze szacowanie kosztów i czasu realizacji projektu.
