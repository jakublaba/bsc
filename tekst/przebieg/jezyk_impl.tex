\section{Język implementacji}
Projekt został zaimplementowany w języku \href{https://www.rust-lang.org/}{Rust}.
Jest to stosunkowo nowa technologia, pierwsze stabilne wydanie tego języka pojawiło się w maju 2015 roku.
Rust reklamuje się przede wszyskim wydajnością i bezpieczeństwem pod kątem pamięci.
Poza osobistą preferencją, język został wybrany ze względu na wygodę implementacji serializacji oraz deserializacji i interfejsu CLI. \\

Przed przystąpieniem do implementacji aplikacji, zadania były tworzone ręcznie, a następnie eksportowane za pomocą Pythona z użyciem biblioteki \href{https://pypi.org/project/jiraone/}{jiraone}.
Biblioteka ta umożliwia szybkie i proste eksporty na projektach w Jirze, oferując filtrowanie eksportowanych danych za pomocą specjalnej składni zapytań Jiry -- \href{https://www.atlassian.com/blog/jira-software/jql-the-most-flexible-way-to-search-jira-14}{JQL}.
