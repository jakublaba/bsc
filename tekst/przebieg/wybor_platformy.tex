\section{Wybór platformy}
W ramach wstępnych badań do pracy zostały porównane funkcjonalności platform:
\begin{itemize}
    \item \href{https://www.atlassian.com/software/jira}{Jira}
    \item \href{https://www.jetbrains.com/youtrack/}{YouTrack}
\end{itemize}
Początkowym pomysłem na realizację projektu było użycie rest api udostępnianego przez obie z platform. \cite[]{JiraApiDocumentation} \cite[]{YouTrackApiDocumentation}

\subsubsection*{Słownik pojęć}
\begin{itemize}
    \item id projektu -- wewnętrzny identyfikator przydzielany projektom przy utworzeniu
    \item klucz projektu -- akronim nazwy projektu lub nazwa projektu skrócona do kilku liter
    \item id zadania -- wewnętrzny identyfikator przydzielany zadaniom przy utworzeniu
    \item klucz zadania -- identyfikator zadań bardziej czytelny dla człowieka, numer zadania w kolejności chronologicznej, prefiksowany kluczem projektu.
    Przykładowo, projekt o kluczu \texttt{AG} będzie posiadał zadania o kluczach \texttt{AG-1, AG-2}, itd.
\end{itemize}

\subsection*{Jira}
\subsubsection*{Autoryzacja}
Jira korzysta w swoim rest api z autoryzacji typu \texttt{Basic Auth}, z emailem użytkownika jako loginem i tokenem dostępowym w roli hasła.
Token należy wygenerować na swoim profilu: \texttt{Profile -> Manage your account -> Security -> Create and manage API tokens}.
\subsubsection*{API -- podgląd zadań}
Należy wykonać zapytanie \texttt{GET} na adres \texttt{https://[instancja jiry]/rest/api/2/issues/[id/key]}
Odpowiedź zawiera kompletny zestaw danych.
\begin{lstlisting}[caption=Wybrane dane z zapytania o pojedyncze zadanie (Jira)]
{
    //...
    "id": "10034",
    "self": "https://jakublaba.atlassian.net/rest/api/2/issue/10034",
    "key": "AG-35",
    "fields": {
        "statuscategorychangedate": "2023-05-25T18:04:12.315+0200",
        "issuetype": {
            "self": "https://jakublaba.atlassian.net/rest/api/2/issuetype/10003",
            "id": "10003",
            "description": "A problem or error.",
            "iconUrl": "https://jakublaba.atlassian.net/rest/api/2/universal_avatar/view/type/issuetype/avatar/10303?size=medium",
            "name": "Bug",
            //...
        },
        //...
        "project": {
            "self": "https://jakublaba.atlassian.net/rest/api/2/project/10000",
            "id": "10000",
            "key": "AG",
            "name": "agilemaster",
            "projectTypeKey": "software",
            //...
        },
        // ...
        "created": "2023-05-25T18:04:12.060+0200",
        // ...
        "updated": "2023-05-25T18:04:12.219+0200",
        "status": {
            "self": "https://jakublaba.atlassian.net/rest/api/2/status/10000",
            "description": "",
            "iconUrl": "https://jakublaba.atlassian.net/",
            "name": "To Do",
            "id": "10000",
            "statusCategory": {
                "self": "https://jakublaba.atlassian.net/rest/api/2/statuscategory/2",
                "id": 2,
                "key": "new",
                "colorName": "blue-gray",
                "name": "To Do"
            }
        },
        //...
    }
}
\end{lstlisting}
\subsubsection*{API -- tworzenie zadań}
Aby utworzyć zadanie należy wykonać zapytanie \texttt{POST} na adres \texttt{https://[instancja jiry]/rest/api/2/issues}.
\begin{lstlisting}[caption=Przykładowe zapytanie do stworzenia zadania (Jira)]
{
    "fields": {
        "project": {
            "key": "AG"
        },
        "summary": "Nazwa zadania",
        "description": "Opis zadania",
        "issuetype":  {
            "name": "Epic"
        }
    }
}
\end{lstlisting}

Jak można zaobserwować z odpowiedzi na zapytanie o dane zadania, istnieje pole \texttt{created} zawierające datę utworzenia zadania.
Dokumentacja nie opisuje możliwości ustawienia tego parametru, jednak spróbowałem włączyć go do poprzedniego zapytania.
\begin{lstlisting}[caption=Próba ustawienia daty stworzenia zadania (Jira)]
    {
        "fields": {
            "project": {
                "key": "AG"
            },
            "summary": "Nazwa zadania",
            "description": "Opis zadania",
            "issuetype":  {
                "name": "Epic"
            }
        },
        "created": "2020-12-12T12:30:00.000+0200"
    }
\end{lstlisting}
Skutkuje to odpowiedzią \texttt{400 Bad Request}.
\begin{lstlisting}[caption=Odpowiedź na próbę ręcznego ustawienia daty utworzenia zadania (Jira)]
{
    "errorMessages": [],
    "errors": {
        "created": "Field 'created' cannot be set. It is not on the appropriate screen, or unknown."
    }
}
\end{lstlisting}
Okazało się, że przy każdej operacji serwer nadaje timestamp, i Jira nie udostępnia możliwości nadpisania żadnego z nich.

\subsection*{YouTrack}
\subsubsection*{Autoryzacja}
YouTrack korzysta w swoim rest api z autoryzacji typu \texttt{Bearer Token}.
Token należy wygenerować na swoim profilu: \texttt{Profile -> Account Security -> New Token}.
\subsubsection*{API -- podgląd zadań}
YouTrack nie pozwala na wykonywanie zapytań z użyciem \textbf{klucza projektu} lub \textbf{kluczy zadań}, stąd trzeba zacząć od zdobycia \textbf{id projektu}.
Należy w tym celu wykonać zapytanie \texttt{GET} na adres \texttt{https://[instancja youtracka]/api/admin/projects}, umieszczając w nim parametry: \texttt{?fields=id,name}
\begin{lstlisting}[caption=Zapytanie o id oraz nazwę wszystkich projektów (YouTrack)]
[
    {
        "name": "agilemaster",
        "id": "0-1",
        "$type": "Project"
    }
]
\end{lstlisting}
Już na tym etapie można zaobserwować pewną niewygodę -- w parametrach zapytania trzeba od razu umieścić nazwy pól które chcemy dostać w odpowiedzi.
Jest to dobre rozwiązanie pod kątem optymalizacji, z takiej koncepcji korzysta np. technologia \href{https://graphql.org/}{GraphQL} aby poprzez przesył jedynie potrzebnych informacji zwiększyć wydajność.
Takie rozwiązanie zwiększa jednak złożoność implementacji, oraz mocno przywiązuje użytkownika api do dokumentacji -- w przypadku jej niekompletności reverse-engineering jest niemożliwy.\\
Efekty tego rozwiązania dają się mocno we znaki już przy wykonywaniu prostych zapytań o dane poszczególnych zadań.
Zapytanie \texttt{GET} na adres \texttt{https://[instancja youtracka]/api/issues/[id]} bez żadnych parametrów nie zwraca praktycznie żadnych danych.
\begin{lstlisting}[caption=Dane z zapytania o pojedyncze zadanie bez parametrów (YouTrack)]
{
    "id": "2-21",
    "$type": "Issue"
}
\end{lstlisting}
\subsubsection*{API -- tworzenie zadań}
Aby utworzyć zadanie należy wykonać zapytanie \texttt{POST} na adres \texttt{https://[instancja youtracka]/api/issues}.
\begin{lstlisting}[caption=Przykładowe zapytanie do stworzenia zadania (YouTrack)]
{  
    "summary": "Proof of concept ticket",  
    "description": "This issue was generated via YouTrack rest api",  
    "project": {  
        "id": "0-1"  
    },  
    "type": {  
        "id": "Epic"  
    }  
}
\end{lstlisting}
Podobnie jak w przypadku Jiry, została podjęta próba ręcznego ustawienia pola \texttt{created}.
\begin{lstlisting}[caption=Próba ustawienia daty stworzenia zadania (YouTrack)]
{
	"summary": "Proof of concept ticket",
	"description": "This issue was generated via YouTrack rest api",
	"project": {
		"id": "0-1"
	},
	"type": {
		"id": "Bug"
	},
	"created": 1000
}
\end{lstlisting}
Api zwraca odpowiedź \texttt{200 Ok}, jednak po podejrzeniu danych utworzonego zadania kolejnym zapytaniem,
możemy zaobserwować że próba ręcznego nadpisania daty została po cichu zignorowana.
\begin{lstlisting}[caption=Dane zadania z "ustawioną" datą (YouTrack)]
{
    "created": 1685149212309,
    "id": "2-40",
    "$type": "Issue"
}
\end{lstlisting}

\subsection*{Import z innego źródła danych}
Na żadnej z platform rest api nie udostępniało funkcjonalności potrzebnych do zasymulowania projektu, a więc konieczne
było znalezienie alternatywy.
Po dalszej analizie dokumentacji obu platform, okazało się że każda z nich udostępnia import danych ze statycznego źródła danych. \cite[]{JiraImportDocumentation} \cite[]{YouTrackImportDocumentation}
Pod kątem tego projektu najistotniejsza była funkcjonalność importu z plików.
W przypadku YouTracka możliwością były formaty \texttt{.csv} oraz \texttt{.xlsx}, Jira umożliwia
import z formatów \texttt{.csv} oraz \texttt{.json}.

Początkowo wybór padł na format \texttt{.csv} z uwagi na to że jest on wspierany przez obie platformy i potencjalnie
mogłoby to umożliwić kompatybilność aplikacji z oboma. Po dalszej analizie dokumentacja okazała się jednak niezadowalająca
pod kątem konfiguracji dat operacji na zadaniach.
W Jirze jednak pliki \texttt{.json} miały identyczną strukturę do odpowiedzi zwracanych z jej rest api, a więc
rozległa dokumentacja w tym zakresie mogła być wykorzystana do analizy ich struktury.\\
Finalną decyzją była platforma Jira oraz generowanie projektów w postaci plików \texttt{.json}.
