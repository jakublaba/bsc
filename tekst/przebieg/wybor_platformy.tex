\section{Wybór platformy}
W ramach wstępnych badań do pracy zostały porównane funkcjonalności platform:
\begin{itemize}
    \item \href{https://www.atlassian.com/software/jira}{Jira}
    \item \href{https://www.jetbrains.com/youtrack/}{YouTrack}
\end{itemize}
Początkowym pomysłem na realizację projektu było użycie rest api udostępnianego przez obie z platform.

\subsubsection*{Słownik pojęć}
\begin{itemize}
    \item id projektu -- wewnętrzny identyfikator przydzielany projektom przy utworzeniu
    \item klucz projektu -- akronim nazwy projektu lub nazwa projektu skrócona do kilku liter
    \item id zadania -- wewnętrzny identyfikator przydzielany zadaniom przy utworzeniu
    \item klucz zadania -- identyfikator zadań bardziej czytelny dla człowieka, numer zadania w kolejności chronologicznej, prefiksowany kluczem projektu.
    Przykładowo, projekt o kluczu \texttt{AG} będzie posiadał zadania o kluczach \texttt{AG-1, AG-2}, itd.
\end{itemize}

\subsection*{YouTrack}
\subsubsection*{Autoryzacja}
YouTrack korzysta w swoim rest api z autoryzacji typu \texttt{Bearer Token}.
Należy wygenerować go na swoim profilu: \texttt{Profile -> Account Security -> New Token}.
\subsection*{API}

\subsection*{Jira}
