\section{Implementacji CLI}
Biblioteka \href{https://docs.rs/clap/latest/clap/}{clap} pozwala na prostą implementację interfejsów typu CLI.
Clap wykorzystuje system makr w Ruście, dzięki czemu wystarczy zdefiniować \texttt{struct} z polami reprezentującymi
dostępne argumenty interfejsu wiersza poleceń, a następnie nałożyć na niego makro \texttt{\#[derive(Parser)]}, uzupełniając niektóre
argumenty posiadające bardziej skomplikowane implementacje własnymi implementacjami parsowania.\\
Na przykładzie tego programu, takie argumenty jak autor czy ilość zadań nie potrzebują dodatkowych implementacji, ponieważ są podstawowymi typami języka.
Jednak takie argumenty jak daty w niestandardowym formacie \texttt{dd-mm-YY} czy typ \texttt{User} wczytywany z pliku wymagają
samodzielnej implementacji.\\
Ponadto, biblioteka potrafi wykryć format komentarzy dokumentujących kod i automatycznie dodać do programu flagę \texttt{--help}, renderującą instrukcję
w postaci tej dokumentacji.\\

\begin{lstlisting}[caption=Instrukcja użycia CLI programu]
PS > agilemaster --help
Usage: agilemaster.exe --name <NAME> --author <PATH> --start <DATE> --end <DATE> --issue-amount <AMOUNT>

Options:
    -n, --name <NAME>            Name of the generated project
    -a, --author <PATH>          Fully qualified name (with path) of json file with user data
    -s, --start <DATE>           Start date of the project (dd-mm-YYYY)
    -e, --end <DATE>             End date of the project (dd-mm-YYYY)
    -i, --issue-amount <AMOUNT>  Amount of issues to generate
    -h, --help                   Print help
    -V, --version                Print version
\end{lstlisting}

\begin{lstlisting}[caption=Przykład użycia programu]
PS > agilemaster `
>> --name test `
>> --author user.json `
>> --start 1-2-2024 `
>> --end 31-3-2024 `
>> --issue-amount 70
[src\main.rs:15:5] &args = Cli {
    name: "test",
    author: User {
        name: "sample_user",
        groups: [],
        active: true,
        email: "user@example.com",
        full_name: "John Smith",
    },
    start: 2024-02-01T00:00:00Z,
    end: 2024-03-31T00:00:00Z,
    issue_amount: 70,
}
\end{lstlisting}
