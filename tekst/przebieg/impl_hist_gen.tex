\section{Implementacja mechanizmu pseudolosowego generowania historii zadań}
Największym wyzwaniem tego projektu był mechanizm generowania historii z zachowaniem pewnej logiki w jej chronologii.
Projekt nie zakłada cofania zadań do poprzednich statusów, a więc sytuacje w których zadanie wykonuje przejście \texttt{In Progress -> Done} wcześniej 
niż przejście \texttt{To Do -> In Progress} nie ma sensu.
Projekt nie zakłada również omijania statusów, a więc aby wygenerować zadanie w statusie \texttt{Done}, nie można zastosować jedynie przejścia
\texttt{To Do -> Done}, tylko trzeba wpierw wygenerować przejście \texttt{To Do -> In Progress}.\\
W tym celu trzeba było wyposażyć generator dat w funkcjonalność generowania daty po wskazanej dacie, ale nadal nie wykraczającej
poza terminy rozpoczęcia i zakończenia projektu.\\
Posiadając tę funkcjonalność, potrzeba jeszcze listy dostępnych statusów w kolejności ich naturalnej progresji.
Jako że domyślną konfiguracją projektów Kanban na Jirze są statusy: "TO DO", "IN PROGRESS", "DONE", to zostały one
zaimplementowane w programie "na sztywno" aby nie nakładać dodatkowej konfiguracji na użytkownika.
