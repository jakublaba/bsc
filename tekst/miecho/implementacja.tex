\section{JQL}
\label{impl:jql}
JQL (Jira Query Language) to język zapytań używany na platformie Jira. JQL pozwala użytkownikom na tworzenie złożonych zapytań w celu wyszukiwania i filtrowania zadań, problemów czy zgłoszeń w systemie Jira. Język ten jest
wszechstronny i elastyczny, umożliwiając tym dokładne określenie kryteriów wyszukiwania.

\subsection{Kluczowe funkcje}
\subsubsection*{Składnia podobna do SQL}
JQL umożliwia definiowanie zapytań w sposób zbliżony do innych języków zapytań, co sprawia, że jest stosunkowo łatwy do zrozumienia dla osób mających doświadczenie z bazami danych.
\subsubsection*{Elastyczność}
JQL pozwala na użycie różnych operatorów do tworzenia precyzyjnych zapytań, które można łączyć za pomocą operatorów logicznych
\subsubsection*{Wsparcie różnych typów danych}
JQL umożliwia wyszukiwanie według różnych typów danych, takich jak liczby, daty, tekst czy wartości logiczne
\subsubsection*{Wyszukiwanie według niestandardowych pól}
Użytkownicy mogą korzystać z JQL do wyszukiwania według pól niestandardowych, co daje dużą elastyczność w dostosowywaniu zapytań do konkretnych potrzeb projektów.
\subsubsection*{Zastosowania praktyczne}
JQL jest powszechnie stosowany do tworzenia filtrów, raportów, dashboardów oraz automatyzacji przepływów pracy w Jira, co czyni go kluczowym narzędziem dla administratorów i użytkowników tej platformy.

\subsection{Przykłady zastosowań}
\begin{itemize}
    \item Wyszukiwanie zadań przypisanych do konkretnego użytkownika: \texttt{assignee = jakublaba}
    \item Wyszukiwanie zadań w określonym statusie: \texttt{status = "Selected for Development"}
    \item Wyszukiwanie zadań utworzonych w określonym przedziale czasowym: \texttt{created >= startOfMonth()}
    \item Wyszukiwanie zadań zawierających określone słowo w opisie: \texttt{description \~{} "bug"}
\end{itemize}

\section{Przebieg implementacji}
Implementacja aplikacji powstała w języku \href{https://www.rust-lang.org/}{Rust}. Jest to stosunkowo nowa technologia, pierwsze stabilne wydanie tego języka pojawiło się w maju 2015 roku.

Główną zaletą Rusta jest koncepcja tzw. \textit{abstrakcji o zerowym koszcie} (Zero Cost Abstraction, Zero Overhead Principle), która pozwala
osiągnąć zadowalający kompromis pomiędzy wydajnością a deklaratywnością kodu. Ten termin oznacza optymalizację na poziomie języka, która sprawia że wykorzystane abstrakcje nie będą posiadać dodatkowego kosztu (obliczeniowego czy pamięciowego) w porównaniu
z optymalną ręczną implementacją takich samych mechanizmów.

Poza samą strukturą języka, która pozwala nie poświęcać czytelności kodu dla wydajności, dodatkowym czynnikiem wpływającym na tę decyzję była dostępność bibliotek do serializacji/deserializacji danych oraz implementacji interfejsów CLI. \cite{RustSerde} \cite{RustClap}

Poza samą implementacją, dużą częścią projektu był również reverse-engineering z pomocą wspomnianych wcześniej skryptów do eksportu danych z Jiry, napisanych w języku \href{https://www.python.org/}{Python}.
Wybór tego języka do tego celu był podyktowany przede wszystkim dostępnością biblioteki \href{https://pypi.org/project/jiraone/}{jiraone} oraz prostotą języka, pozwalającą na ich szybkie tworzenie.

Po analizie struktury eksportowanych projektów, możliwe było stworzenie modelu danych.

Rust nie wspiera dziedziczenia, oferuje alternatywę w postaci makra \texttt{\#[derive]}. Makro to przyjmuje jako parametry
nazwy interfejsów, następnie szuka ich implementacji na wszystkich polach struktury, i na ich podstawie implementuje je dla struktury jako
całości. W przypadku braku domyślnej implementacji interfejsu dla danego typu wymagana jest samodzielna implementacja.

Biblioteka do serializacji/deserializacji opiera się w głównej mierze właśnie na tym makrze, w połączeniu z interfejsami \texttt{Serialize} oraz \texttt{Deserialize}.
Z tego powodu konieczna była własna implementacja serializacji dla dat.

Jako że Kanbanowe metryki działają z dokładnością do dni, przyjęte zostało uproszczenie polegające na zignorowaniu godziny.
Pozwala to na uproszczenie interfejsu użytkownika -- konieczne jest podanie jedynie daty w formacie \texttt{dd-mm-YYYY}, a arbitralna
godzina jest potem doklejana przez program.

Kolejnym wyzwaniem była implementacja generatora dat. Daty musiały być pseudolosowe oraz mieszczące się w konkretnym przedziale czasowym
podanym przez użytkownika (data rozpoczęcia oraz zakończenia projeku). Ponadto, do generowania historii przejść zadań pomiędzy statusami
konieczna była funkcjonalność generowania daty po wybranej dacie, nadal mieszczącej się jednak w przedziale.

Rust umożliwia przeciążanie operatorów, co skutkuje bardzo czytelnym kodem operującym na bardziej złożonych typach.
Dzięki temu na datach można w bardzo prosty sposób wykonywać dodawanie i odejmowanie wybranych przedziałów czasowych czy porównywanie.
Dzięki dokładności metryk jedynie do dni, można przeprowadzić generowanie liczb pseudolosowych w zbiorze liczb całkowitych.

Konieczne było stworzenie mechanizmu generowania historii zadań z zachowaniem pewnej logiki w chronologii.
Projekt nie zakłada cofania zadań do poprzednich statusów, a więc sytuacja w której przejście \texttt{In Progress -> Done} następuje
wcześniej niż przejście \texttt{To Do -> In Progress} nie ma sensu.

Projekt nie zakłada również omijania statusów, więc żeby wygenerować zakończone zadanie, nie można wykonać przejścia \texttt{To Do -> Done},
trzeba zamiast tego wygenerować zestaw przejść prowadzący do tego statusu, zachowując cały czas chronologiczną kolejność w datach
tych operacji. Właśnie w tym celu potrzebna była funkcjonalność generowania dat po wybranej dacie, ale nadal mieszczącej się w przedziale trwania projektu.
Do implementacji takiego rozwiązania potrzebna jest również lista przejść w kolejności ich naturalnej progresji.

Mimo że zakres pracy tego nie obejmował, przydatną funkcją generatora byłaby możliwość generowania metadanych zadań potrzebnych analitycznym wtyczkom Jiry, w celu generowania raportów bezpośrednio na platformie.
Jako że dokumentacja nie opisuje szczegółów implementacyjnych tych narzędzi, konieczne było uciec się znów do technik reverse-engineering.

W tym celu w testowym projekcie zostało stworzone i poprzeciąganie do różnych statusów kilku zadań. Po eksporcie za pomocą wcześniej przygotowanych skryptów można było zaobserwować w zadaniach ciekawe pole niestandardowe.
\newpage
\begin{lstlisting}[caption=Pole niestandardowe "time in status"]
{
    "fieldName": "[CHART] Time in Status",
    "fieldType": "com.atlassian.jira.ext.charting:timeinstatus",
    "value": "10030_*:*_1_*:*_0_*|*_10028_*:*_1_*:*_16052"
}
\end{lstlisting}
Struktura wartości tego pola wygląda następująco: składa sie ono z szerego rekordów reprezentujących dane związane z kolejnymi statusami w których znajdowało się zadanie, rozdzielone za pomocą \lstinline!_*|*_!.
Dane w obrębie każdego rekordu są separowane za pomocą \lstinline!_*:*_!. Każdy rekord składa się z 3 danych: id statusu, ile razy zadanie znalazło się w tym statusie, oraz całkowita ilość czasu spędzona w nim, wyrażona w milisekundach.
Id statusów można zdobyć jedynie za pośrednictwem rest api i są one unikalne na każdej instancji Jiry.

Wymaganie wprowadzenia tych danych przez użytkownika znacznie skomplikowałoby korzystanie z aplikacji, a więc konieczna była implementacja
pobierania odpowiednich danych z rest api na podstawie nazw statusów wprowadzanych przez użytkownika. Po implementacji generowania tego pola, wykresy na Jirze jednak nadal nie generowały się poprawnie -- pomimo widoczności poprawnych statusów
zadań we wszystkich innych miejscach w systemie, na potrzeby generowania statusów wszystkie zadania były traktowane jakby nadal znajdowały się w pierwszym statusie ("To Do" lub "Backlog" w zależności od testowanych szablonów)

Ostatnim krokiem w implementacji projektu była implementacja interfejsu użytkownika. Sama aplikacja jest na tyle prosta, że GUI wprowadziłoby więcej
złożoności niż jest to potrzebne, a więc zaimplementowane zostało CLI.

\begin{lstlisting}[caption=CLI programu agilemaster]
PS C:\Users\Kuba\RustroverProjects\agilemaster> agilemaster --help
Usage: agilemaster.exe [OPTIONS] --name <NAME> --author <PATH> --auth-params <PATH> --start <DATE> --end <DATE> --issue-amount <AMOUNT>

Options:
    -n, --name <NAME>             Name of the generated project
    -a, --author <PATH>           Fully qualified name (with path) of json file with user data
    -a, --auth <PATH>             Fully qualified name (with path) of json file with authentication data
    -s, --start <DATE>            Start date of the project (dd-mm-YYYY)
    -e, --end <DATE>              End date of the project (dd-mm-YYYY)
    -i, --issue-amount <AMOUNT>   Amount of issues to generate
    -s, --statuses <STATUSES>...  Space-separated list of statuses available in project
    -h, --help                    Print help
    -V, --version                 Print version
\end{lstlisting}

Implementacja flag biorących jako argumenty tekst lub liczby nie była problemem, biblioteka do implementacji CLI również opiera się na makrze
\texttt{\#[derive]}, w tym przypadku w połączeniu z interfejsem \texttt{Parser}.
Konieczna była ręczna implementacja parsowania dat, z uwagi na wspomniane wcześniej uproszczenie dla użytkownika zwalniające go z podawania godziny oraz strefy czasowej.
Kolejną daną niezbędną do wprowadzenia są dane użytkownika widniejącego jako twórca projektu w Jirze.

Trzeba tutaj podać login, email, listę grup, czy konto jest aktywne oraz pełne imię i nazwisko wprowadzone na profilu platformy.
Rozbicie tych danych na oddzielne flagi znacznie zmniejszyłoby wygodę korzystania z aplikacji, a więc do podania danych użytkownika konieczne jest stworzenie pliku \texttt{.json}.
Dla zwiększenia komfortu korzystania z aplikacji, plik posiada identyczną strukturę do danych eksportowanych z platformy Jira.
\begin{lstlisting}[caption=Przykładowy plik z danymi użytkownika]
{
    "name": "jakublaba",
    "groups": [
        "Administrator",
        "Member",
        "Viewer"
    ],
    "active": true,
    "email": "jakub.maciej.laba@gmail.com",
    "fullname": "Jakub Łaba"
}
\end{lstlisting}

Program korzysta również z rest api Jiry na potrzeby zidentyfikowania parametrów różniących się na każdej instancji Jiry, aby odciążyć użytkownika z dodatkowej konfiguracji.
Wymagane są jednak dane dostępowe do api w postaci kolejnego pliku \texttt{.json}
\begin{lstlisting}[caption=Przykładowy plik z danymi dostępowymi do api Jiry]
{
  "email": "jakub.maciej.laba@gmail.com",
  "apiKey": "secret-key",
  "url": "https://jakublaba.atlassian.net"
}
\end{lstlisting}
