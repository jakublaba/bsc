\section{Przegląd i wybór platform do analizy}
Na rynku istnieje wiele platform oferujących narzędzia do zarządzania projektami. Do przeglądu zostały wybrane platformy umożliwiające realizację projektów w metodyce Kanban, z uwagi na wybrane we wcześniejszej
części pracy metryki.

Przegląd krótko charakteryzuje każdą z platform oraz skupia się na wyróżniających cechach oraz obecności darmowego planu.
\begin{itemize}
    \item \href{https://www.atlassian.com/software/jira}{Jira}
    \item \href{https://azure.microsoft.com/en-us/products/devops/}{Azure DevOps}
    \item \href{https://www.jetbrains.com/youtrack/}{YouTrack}
    \item \href{https://asana.com/}{Asana}
    \item \href{https://businessmap.io/}{Businessmap (wcześniej Kanbanize)}
\end{itemize}

\subsection{Jira}
\subsubsection*{Charakterystyka}
Rozwiązanie oferowane przez firmę Atlassian od lat jest jednym z najpopularniejszych narzędzi do zarządzania projektami, szczególnie w środowiskach IT. Oferuje szereg funkcji, które sprawiają że jest uniwersalnym
narzędziem, umożliwiającym zarządzanie różnorodnymi projektami -- od oprogramowania i innych kategorii IT po zespoły biznesowe, marketingowe czy HR.

Jira jest znana z elastyczności, pozwalając na dostosowywanie przepływów pracy, tworzenie automatyzacji oraz integracje z innymi narzędziami Atlassian, takimi jak Confluence i Bitbucket.

Jira oferuje również wsparcie dla różnych metodyk zarządzania projektami, w tym Scrum i Kanban, co czyni ją odpowiednią zarówno dla projektów krótkoterminowych, jak i długoterminowych. Jej system śledzenia błędów i problemów (issue tracking)
jest jednym z najbardziej rozbudowanych na rynku, a użytkownicy mogą tworzyć zaawansowane raporty, korzystając z wbudowanych narzędzi do analizy danych.

Jira posiada również specjalny język zapytań -- Jira Query Language (JQL), pozwalający na bardzo zaawansowane funkcjonalności dotyczące wyszukiwania i filtrowania danych.
JQL dokładniej opisywany jest w rozdziale implementacyjnym \ref{impl:jql}.
\subsubsection*{Wyróżniające cechy}
\begin{itemize}
    \item Silne wsparcie dla metodyk zwinnych, w tym Scrum i Kanban
    \item Zaawansowane możliwości automatyzacji i dostosowywania przepływów pracy
    \item Integracja z innymi narzędziami Atlassian, co umożliwia łatwe zarządzanie dokumentacją i kontrolę wersji.
    \item JQL
\end{itemize}
\subsection*{Darmowy Plan}
Jira oferuje darmowy plan dla małych zespołów i użytku indywidualnego. W darmowym planie istnieją ograniczenia dotyczące liczby użytkowników i projektów, ale jest to wystarczające dla mniejszych zespołów, którzy chcą wypróbować platformę lub zarządzać niewielką liczbą projektów.

\subsection{Azure DevOps}
\subsubsection*{Charakterystyka}
Azure DevOps to produkt firmy Microsoft, będący częścią większego ekosystemu usług Azure.

Platforma ta została stworzona z myślą o zespołach technicznych i oferuje zestaw narzędzi do zarządzania projektami, kontroli wersji, CI/CD oraz monitorowania procesów. Azure DevOps jest zaprojektowany do integracji
z innymi usługami Azure, co pozwala na tworzenie kompleksowych rozwiązań chmurowych.

Azure DevOps skupia się na dostarczaniu narzędzi dla zespołów programistycznych umożliwiając tworzenie zautomatyzowanych przepływów pracy, śledzenia zadań i zarządzania repozytoriami kodu. Dzięki integracji z Visual Studio,
platforma umożliwia programistom płynne przechodzenie między różnymi etapami procesu tworzenia oprogramowania.
\subsubsection*{Wyróżniające cechy}
\begin{itemize}
    \item Silna orientacja na zespoły techniczne, z naciskiem na kontrolę wersji i CI/CD
    \item Integracja z platformą Azure i narzędziami Microsoft, takimi jak Visual Studio
    \item Możliwość zarządzania całym procesem tworzenia oprogramowania, od programowania po wdrażanie i monitorowanie
\end{itemize}
\subsection*{Darmowy Plan}
Azure DevOps oferuje darmowy plan dla małych zespołów i użytku indywidualnego. Darmowy plan obejmuje ograniczenia dotyczące liczby użytkowników, ale zapewnia dostęp do większości funkcji platformy, co czyni go atrakcyjnym 
dla małych zespołów i projektów testowych.

\subsection{YouTrack}
\subsubsection*{Charakterystyka}
YouTrack to rozwiązanie od JetBrains, firmy znanej z szerokiej oferty zintegrowanych środowisk programistycznych (IDE). Platforma ta jest zaprojektowana z myślą o prostocie i elastyczności, z naciskiem na produktywność. YouTrack oferuje
funkcje, które ułatwiają zarządzanie projektami w metodyce Kanban, z możliwością szybkiego wyszukiwania zadań i elastycznego dostosowywania interfejsu.

YouTrack wyróżnia się intuicyjnym interfejsem, który jest mniej skomplikowany niż inne platformy, co ułatwia nowym użytkownikom szybką naukę i rozpoczęcie pracy. Mimo prostoty, platforma zapewnia szereg zaawansowanych funkcji,
takich jak automatyzacja, integracja z innymi narzędziami JetBrains i obsługa metodyk zwinnych.
\subsubsection*{Wyróżniające cechy}
\begin{itemize}
    \item Silna orientacja na zespoły techniczne dzięki integracji z popularnymi IDE JetBrains
    \item Przejrzysty interfejs i szybka nawigacja platformy przy użyciu klawiatury
    \item Możliwość dostosowywania przepływów pracy i łatwej automatyzacji zadań
\end{itemize}
\subsection*{Darmowy Plan}
YouTrack oferuje darmowy plan dla małych zespołów i użytku indywidualnego. Darmowy plan pozwala na ograniczoną liczbę użytkowników i projektów, ale zawiera większość funkcji dostępnych w płatnych planach.

\subsection{Asana}
\subsubsection*{Charakterystyka}
Asana to narzędzie do zarządzania projektami, które skierowane jest przede wszystkim do zespołów biznesowych i kreatywnych. Oferuje przejrzysty i intuicyjny interfejs, który umożliwia zarządzanie projektami w różnych formach,
takich jak tablice Kanban, osie czasu czy kalendarze. Asana kładzie nacisk na współpracę i umożliwia zespołom organizowanie pracy w sposób, który najlepiej odpowiada ich potrzebom.

Asana jest elastyczna i pozwala użytkownikom dostosować przepływy pracy oraz tworzyć zaawansowane struktury projektów. Narzędzie to jest często używane przez zespoły kreatywne, marketingowe i biznesowe, ale znajduje również
zastosowanie w innych obszarach, gdzie ważna jest przejrzystość i współpraca.
\subsubsection*{Wyróżniające cechy}
\begin{itemize}
    \item Wielokrotne widoki projektów (tablica Kanban, oś czasu, kalendarz), co pozwala na dostosowanie platformy do różnych potrzeb
    \item Silne wsparcie dla współpracy między zespołami i integracji z innymi narzędziami
    \item Możliwość łączenia projektów w większe struktury oraz tworzenia raportów
\end{itemize}
\subsection*{Darmowy Plan}
Asana oferuje darmowy plan dla małych zespołów i użytku indywidualnego. Darmowy plan ma ograniczenia dotyczące liczby użytkowników i projektów, ale zawiera podstawowe funkcje platformy
i jest odpowiedni dla małych zespołów lub osób chcących przetestować Asanę.

\subsection{Businessmap (wcześniej Kanbanize)}
\subsubsection*{Charakterystyka}
Businessmap, znane wcześniej jako Kanbanize, to platforma skoncentrowana na metodyce Kanban. Jest zaprojektowana z myślą i wizualizacji pracy i przepływu zadań. Centralnym elementem platformy są tablice Kanban, które umożliwiają
śledzenie postępów zadań i zarządzanie przepływem pracy. Platforma ta oferuje również narzędzia do automatyzacji procesów oraz śledzenia aktywności zespołów.

Businessmap jest mniej skomplikowane niż niektóre inne platformy do zarządzania projektami, ale zapewnia wszystkie niezbędne narzędzie do efektywnego zarządzania projektami w stylu Kanban. Dzięki swojej specjalizacji, platforma
ta jest często wybierana przez zespoły chcące skupić się na metodyce Kanban bez konieczności korzystania z bardziej skomplikowanych narzędzi.
\subsubsection*{Wyróżniające cechy}
\begin{itemize}
    \item Całkowita orientacja na metodyce Kanban, co czyni platformę intuicyjną dla zespołów, które preferują ten styl pracy
    \item Narzędzia do wizualizacji pracy i przepływu zadań, które umożliwiają śledzenie postępów projektów w czasie rzeczywistym
    \item Możliwość automatyzacji procesów i śledzenia efektywności pracy zespołów
\end{itemize}
\subsection*{Darmowy Plan}
Businessmap nie oferuje darmowego planu, jedynie 90-dniowy okres próbny.

\subsection*{Wybór platform do dalszej analizy}
Wybierając platformy do dalszej analizy, brany były pod uwagę takie czynniki jak dostępność darmowych planów, doświadczenie z użytkowaniem oraz kompatybilność z metodyką Kanban. W związku z tym pomimo interfesującej orientacji
Businessmap na Kanbana, platforma ta nie oferuje darmowego planu, co utrudnia jej pełną ocenę bez angażowania dodatkowych zasobów finansowych. Ponadto darmowy okres próbny Businessmap, choć długi (3 miesiące), w przypadku implementacji
symulacji w oparciu o tę platformę, nadal wymagałby finalnej subskrypcji, co mogłoby być ograniczeniem dla niektórych użytkowników.

Dlatego zdecydowałem skupić się na innych platformach, które są dostępne w darmowej wersji oraz mają uznaną reputację w dziedzinie zarządzania projektami: Jira oraz YouTrack. Obie platformy oferują darmowe plany, co pozwala na pełną ocenę
funkcjonalności bez dodatkowych zobowiązań finansowych.

Oprócz dostępności darmowych planów, wybór Jiry i YouTracka wynika również z doświadczenia, które już posiadam w korzystaniu z tych platform. Dzięki temu możliwe jest przeprowadzenie bardziej wnikliwej analizy, wykorzystując praktyczną
wiedzę i podstawową znajomość ich funkcjonalności. To z kolei przekłada się na bardziej miarodajną ocenę wad i zalet każdej z nich oraz porównanie ich możliwości w kontekście zastosowania do symulacji projektów.

W przypadku Azure DevOps oraz Asany, choć są to uznane platformy, to nie posiadam tak dobrego doświadczenia w ich używaniu, co mogłoby wpłynąć na dokładność analizy, przez co ich ocena mogłaby być mniej rzetelna od oceny Jiry i YouTracka.

\section{Badanie możliwości REST API}
W początkowej fazie pracy, projekt miał opierać się o rest api, udostępniane przez obie platformy. \cite{JiraApiDocumentation} \cite{YouTrackApiDocumentation}
W tym podrozdziale pracy często wykorzystywane są pojęcia mogące brzmieć podobnie, warto zacząć więc od zdefiniowania prostego słownika:
\begin{itemize}
    \item id projektu -- wewnętrzny identyfikator przydzielany projektom przy utworzeniu przez system
    \item klucz projektu -- akronim nazwy projektu lub nazwa projektu skrócona do kilku liter
    \item id zadania -- wewnętrzny identyfikator przydzielany zadaniom przy utworzeniu przez system
    \item klucz zadania -- identyfikator zadań bardziej czytelny dla człowieka, numer zadania w kolejności chronologicznej, prefiksowany kluczem projektu.
    Przykładowo, projekt o kluczu \texttt{AG} będzie posiadał zadania o kluczach \texttt{AG-1, AG-2}, itd.
\end{itemize}

W celu pobrania danych zadania należy wykonać zapytanie \texttt{GET} odpowiednio na adres:
\begin{itemize}
    \item \texttt{https://[instancja jiry]/rest/api/2/issues/[id/key]} -- Jira
    \item \texttt{https://[instancja youtracka]/api/issues/[id]} -- YouTrack
\end{itemize}

W przypadku Jiry odpowiedź zawiera kompletny zestaw danych -- poza podstawowym informacjami jak id, klucz czy projekt do którego należy zadanie, otrzymujemy również ogromną ilość
danych związanych z historią zadania, możliwościami operacji na zadaniu, a nawet metadanych związanych z szablonami wykorzystanymi do tworzenia projektu lub użytych rozszerzeń.

YouTrack domyślnie zwraca bardzo okrojony zestaw danych -- jedynie typ projektu. "Rest" api YouTracka nie spełnia założeń architektury REST \cite{RoyTFieldingRest}, strukturą i sposobem korzystania przypomina 
bardziej takie rozwiązanie jak \href{https://graphql.org/}{GraphQL}. Wszystkie pola, które chcemy uzyskać w odpowiedzi musimy umieścić w parametrach zapytania.

Przykładowo, aby uzyskać nazwę zadania oraz datę jego utworzenia, należy dodać parametr \texttt{?fields=name,created}. Jest to dobre rozwiązanie pod kątem optymalizacji liczby zapytań oraz zużycia zasobów, jednak
w znaczący sposób komplikuje ono użytkowanie api, bardzo mocno przywiązując użytkownika do dokumentacji w celu odnalezienia informacji o dostępnych danych. W przypadku niekompletności dokumentacji reverse-engineering jest niemożliwy.

Dodatkową wadą YouTracka w porównaniu do Jiry była niemożliwość identyfikacji zadań w zapytaniach po ich kluczach. Konieczne było uzyskanie ich id, co można było osiągnąć de facto tylko przez każdorazowe dodatkowe zapytanie o projekt z listą zadań, w której z kolei
również trzeba było sprecyzować aby zwrócić ich id w parametrach zapytania. Już ta jedna różnica na starcie zasadniczo przechyliła szalę na korzyść Jiry.

Najistotniejsze z punktu widzenia projektu były tutaj dane związane z historią zadań, konkretnie historią przejść pomiędzy statusami.
Informacje potrzebne do realizacji projektu to przede wszystkim daty -- rozpoczęcia i zakończenia zadania (jeżeli zostało już zakończone), oraz wszystkich przejść pomiędzy statusami, z informacją
z jakiego do jakiego statusu przeszło w danym momencie. Jedyne dane z tego zakresu jakie można było znaleźć w odpowiedzi to daty utworzenia zadania i jego ostatniej aktualizacji.

W przypadku obu serwisów dokumentacja api nie opisuje danych stricte związanych z taką historią, jedynie szczątkowe dane takie jak data utworzenia, zamknięcia czy ostatniej aktualizacji (tj. dowolnej operacji związanej z konkretnym zadaniem).
Przyjmując uproszczenie poprzez postawienie znaku równości pomiędzy datą utworzenia zadania i datą jego rozpoczęcia, nadal nie można by przeanalizować efektywności przepływu, a analiza reszty metryk byłaby dość przybliżona.
Moment utworzenia zadania nie jest jednoznaczny z rozpoczęciem pracy nad nim, ponieważ po utworzeniu nadal może ono czekać przez pewien czas w backlogu zanim osiągnie punkt zobowiązania.

W celu utworzenie zadania należy wykonać zapytanie \texttt{POST} odpowiednio na adres:
\begin{itemize}
    \item \texttt{https://[instancja jiry]/rest/api/2/issues} -- Jira
    \item \texttt{https://[instancja youtracka]/api/issues} -- YouTrack
\end{itemize}
W przypadku pomyślnego utworzenia zadania, api obu platform zwraca w odpowiedzi datę utworzenia zadania. Jira korzysta ze standardu ISO 8601, natomiast YouTrack z formatu Unix Time \cite{UnixProgrammersManual}.

Dokumentacje obu platform nie wspominają możliwości ustawienia tego pola. W przypadku Jiry próba kończy się odpowiedzią \texttt{400 Bad Request}, a YouTrack po prostu po cichu ignoruje tę daną
w ciele zapytania.

Na tym etapie praca nad projektem została mocno spowolniona, ponieważ żadne z api nie udostępnia ani wszystkich potrzebnych danych do kompletnej analizy metryk, ani nie pozwala nawet ich ręcznie wprowadzić -- daty każdej operacji
są automatycznie nadawane przez serwer i api nie pozwala w żaden sposób na obejście tego mechanizmu.

\section{Podsumowanie i wybór platformy do implementacji rozwiązania}
W związku z brakiem potrzebnych do realizacji projektu funkcjonalności w rest api obydwu platform, konieczne było poszukiwanie alternatywnego rozwiązania.
YouTrack nie oferuje tutaj żadnych specjalnych funkcjonalności, natomiast Jira posiada swój własny język do tworzenia zapytań -- JQL (Jira Query Language). \cite{YouTrackSearch} \cite{JiraJQL}

Na tym etapie projekt skupił się w całości na platformie Jira. Zapoznanie się z podstawami JQL było inwestycją czasu, która zwróciła się z nawiązką, szczególnie kiedy znalazłem również bibliotekę \href{https://pypi.org/project/jiraone/}{jiraone} do języka Python,
która pozwala na eksport danych uzyskanych za pomocą JQL do plików \texttt{.json}. Skrypt do takiego eksportu danych zdecydowanie przyspieszył ich zdobywanie, oraz uprościł przechowywanie ich do późniejszej analizy.

Dane zdobyte za pomocą JQL okazały się bardziej kompletne -- tutaj już można było odnaleźć informacje o historii poszczególnych zadań. Analiza danych pozyskana w ten sposób była również prostsza, z uwagi na możliwość nałożenia filtrów i zawężenia ich
jedynie do tych istotnych na dany moment.

Po dalszych badaniach udało znaleźć się alternatywny sposób wypełniania projektów zadaniami -- narzędzie do migracji danych z zewnętrznego źródła do Jiry. \cite{JiraImportExport}
Jira pozwala importować dane z różnych źródeł, ale szczególną uwagę przykuły pliki \texttt{.csv} oraz \texttt{.json} ze względu na prostotę.

Dokumentacja importu z obu formatów okazała się dość wybrakowana pod kątem danych istotnych dla projektu. Jednak jako że eksport za pomocą JQL zwracał format JSON, reverse-engineering za pomocą wcześniej napisanych skryptów w Pythonie nie był
żadnym problemem.
