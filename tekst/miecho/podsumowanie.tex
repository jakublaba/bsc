Celem pracy "Generowanie danych na potrzeby nauki metryk w inżynierii oprogramowania" było stworzenie rozwiązania umożliwiającego symulację projektów z użyciem popularnych platform w kontrolowanym środowisku. Dzięki temu
zespoły oraz osoby uczące się mogłyby doskonalić umiejętność analizy i interpretacji metryk bez konieczności pracy w rzeczywistymi danymi, które mogą być poufne lub po prostu trudne do zdobycia. W rezultacie powstała aplikacja
w formie programu CLI, umożliwiająca generowanie danych do symulacji projektów prowadzonych w metodyce Kanban na platformie Jira.

Prace rozpoczęły się od analizy różnych metryk stosowanych stosowanych do pomiaru efektywności zespołów w kontekście inżynierii oprogramowania. Przeprowadzony został również przegląd platform do zarządzania projektami, ze szczególnym
uwzględnieniem Jiry i YouTracka, skupiając się na metrykach związanych z metodyką Kanban.

Kolejnym krokiem była analiza struktury danych projektów i zadań, opierając się na dokumentacji oraz technikach reverse-engineering. To pozwoliło na zdefiniowanie podstawowych mechanizmów, kluczowych dla implementacji rozwiązania.

Aplikacja agilemaster pozwala na szybkie generowanie danych projektowych przy minimalnej konfiguracji. Aplikacja umożliwia wybór okna czasowego w którym mają być generowane zadania oraz wprowadzenie dowolnych statusów zadań, co pozwala
na elastyczność pod kątem scenariuszy lub wymagań projektów tworzonych z różnych szablonów lub według niestandardowych spersonalizowanych konfiguracji. Dzięki temu zestawy danych do kompleksowej analizy, składające się z setek bądź tysięcy
zadań, są na wycięgnięcie ręki w przeciągu zaledwie kilku sekund.

Rozwiązanie może być niezwykle wartościowe dla celów szkoleniowych, ponieważ umożliwia naukę w środowisku, które jest zgodne z rzeczywistymi warunkami komercyjnymi. Osoby uczące się analizy metryk i zarządzania projektami w metodykach
zwinnych mogą teraz ćwiczyć i eksperymentować z różnymi scenariuszami, zyskując cenne doświadczenie.

Poprzez dostarczenie praktycznego narzędzia, znacznie usprawniającego proces nauki analizy metryk, praca może przyczynić się do popularyzacji wiedzy niosącej za sobą znaczącą wartość biznesową.
