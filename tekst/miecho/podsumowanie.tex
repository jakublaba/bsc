W pracy została podjęta analiza szeregu metryk pod kątem ich zastosowania w kontekście pomiaru pracy zespołu.
Przeprowadzono przegląd platform do zarządzania projektami, dalej dokonując bardziej szczegółowej analizy Jiry oraz YouTracka pod kątem
metryk metodyki Kanban.
Następnie przeprowadzana została analiza struktury danych projektu i zadań na podstawie dokumentacji oraz technik reverse-engineering, oraz zaimplementowane rozwiązanie
pozwalające na generowanie przykładowych danych, wybierając czas trwania projektu oraz konfigurując dostępne w projekcie statusy.
Dane uzyskane w ten sposób pozwoliły na wypełnienie projektu danymi oraz zapewnienie chronologii w historii zadań.

Aplikacja agilemaster pozwala na szybkie wypełnianie projektów danymi, wymagając przy tym minimalnej konfiguracji. Dzięki temu dla celów dydaktycznych można w zaledwie kilka sekund wygenerować
zestawy danych zawierające setki lub tysiące zadań do kompleksowej analizy. Zminimalizowanie wysiłku związanego z przygotowywaniem ćwiczeniowych zestawów danych bezpośrednio w środowisku
mającym szerokie zastosowanie komercyjne pozwala na popularyzację bardzo wartościowej wiedzy z zakresu zarządzania projektami w metodykach zwinnych.

Niestety nie udało się osiągnąć integracji z narzędziami do generowania wykresów bezpośrednio w Jirze, z uwagi na brak wystarczającej dokumentacji.
Z analizy danych eksportowanych z Jiry zakładałem, że opierają się one na niestandardowym polu "timespent", które opisuje ile razy zadanie znajdowało się w każdym ze statusów
oraz ile łącznie spędziło w nim czasu. Po implementacji tego pola jednak nadal wszystkie zadania na wykresach były kategoryzowane jakby nadal znajdowały się w pierwszym statusie (w zależności
od testowanych szablonów "To Do" lub "Backlog").

\section*{Możliwości dalszej rozbudowy projektu}
W obecnej formie aplikacja posiada jedynie podstawowe funkcjonalności, pozostawiając nadal wiele pola do rozwoju.
Istotnym ulepszeniem byłaby wspomniana integracja z narzędziami do generowania wykresów wbudowanymi bezpośrednio w Jirę aby zwiększyć wygodę analizy generowanych danych.
Wiele wartości dodałaby tutaj możliwość dokładniejszej kontroli nad parametrami generowania, aby móc tworzyć bardziej precyzyjne scenariusze do nauki, pozwalając np. na celowe generowanie projektów
które postępują bardzo sprawnie lub wręcz przeciwnie. Do osiągnięcia takich rezultatów można by było wprowadzić kontrolę nad rozkładem zadań w różnych statusach -- konfigurując precyzyjne ilości lub
ich proporcje do siebie nawzajem. Innym parametrem którego kontrolowanie pozwoliłoby zwiększyć atrakcyjność aplikacji jest możliwość manipulacji czasem trwania zadań lub czas ich przebywania w 
poszczególnych statusach -- w sposób precyzjny, lub na podstawie wybranych rozkładów losowych.
Jeśli chodzi o kompatybilność aplikacji z więcej niż jedną platformą, jest to poniekąd osiągnięte poprzez wybór Jiry, będącej najpopularniejszą z nich.
Popularnością Jiry podyktowana jest możliwość importu projektów z niej w innych narzędziach, takich jak YouTrack czy Azure DevOps.
