W obecnej formie aplikacja posiada jedynie podstawowe funkcjonalności, pozostawiając nadal wiele pola do rozwoju.

Istotnym ulepszeniem byłaby wspomniana integracja z narzędziami do generowania wykresów wbudowanymi bezpośrednio w Jirę aby zwiększyć wygodę analizy generowanych danych.

Wiele wartości dodałaby tutaj możliwość dokładniejszej kontroli nad parametrami generowania, aby móc tworzyć bardziej precyzyjne scenariusze do nauki, pozwalając np. na celowe generowanie projektów
które postępują bardzo sprawnie lub wręcz przeciwnie. Do osiągnięcia takich rezultatów można by było wprowadzić kontrolę nad rozkładem zadań w różnych statusach -- konfigurując precyzyjne ilości lub
ich proporcje do siebie nawzajem. Innym parametrem którego kontrolowanie pozwoliłoby zwiększyć atrakcyjność aplikacji jest możliwość manipulacji czasem trwania zadań lub czas ich przebywania w 
poszczególnych statusach -- w sposób precyzjny, lub na podstawie wybranych rozkładów losowych.

Jeśli chodzi o kompatybilność aplikacji z więcej niż jedną platformą, jest to poniekąd osiągnięte poprzez wybór Jiry, będącej najpopularniejszą z nich.
Popularnością Jiry podyktowana jest możliwość importu projektów z niej w innych narzędziach, takich jak YouTrack czy Azure DevOps.