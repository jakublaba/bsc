\section*{Cel pracy}
Praca "Generowanie danych na potrzeby nauki metryk w inżynierii oprogramowania" ma na celu analizę różnych metryk w inżynierii oprogramowania, przegląd platform pod kątem funkcjonalności umożliwiających
ich symulację, przebadanie struktur danych w projekcie oraz implementację rozwiązania oraz integrację z wybraną platformą.
Założeniem projektu jest umożliwienie użytkownikowi wprowadzenia okresu trwania projektu w postaci dat jego rozpoczęcia i zakończenia, oraz liczby zadań do wygenerowania.
W zależności od wyboru platformy konieczne może być wprowadzenie dodatkowych parametrów, takich jak dane dostępowe do platformy lub dodatkowe dane związane z użytkownikiem.
Projekt zakłada kompatybilność z projektami przeprowadzanymi w metodyce Kanban.
