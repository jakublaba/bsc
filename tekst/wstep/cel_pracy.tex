\section*{Cel pracy}
Celem pracy "Generowanie danych na potrzeby nauki metryk w inżynierii oprogramowania" było stworzenia rozwiązania umożliwiającego symulację projektów w jednym z popularnych środowisk do zarządzania projektami,
co umożliwi naukę analizy metryk bez potrzeby uzyskiwania dostępu do prawdziwych projektów.

Obecnie, aby przeprowadzić szkolenie z zakresu analizy metryk w inżynierii oprogramowania z wykorzystaniem jednej z najszerzej stosowanych platform, w przypadku braku dostępu do danych prawdziwego projektu,
konieczne jest jego ręczne spreparowanie.

Jest to możliwe do osiągnięcia, jednak ilość danych jaką należy wprowadzić w celu umożliwienia miarodajnej analizy sprawia, że proces ten robi się ekstremalnie czasochłonny, żmudny i podatny na błędy.
Automatyzacja procesu wypełniania projektów relatywnie realistycznymi parametrami pozwoliłaby upowszechnić dostęp do wiedzy o analizie metryk w inżynierii oprogramowania.
Analiza metryk jest o tyle istotna, że pozwala przeprowadzać uwielbianą przez biznes ocenę produktywności w obiektywny sposób.

Praca intelektualna nie jest prosta do zmierzenia, stąd niezbędny jest odpowiedni poziom ekspertyzyaby w ogóle poprawnie dobrać metryki, nie mówiąc już o wyciąganiu z nich rozsądnych wniosków.
"Generowanie danych na potrzeby nauki metryk w inżynierii oprogramowania" kładzie nacisk na popularyzację takiej wiedzy właśnie w tym celu.

Projekt zakłada możliwość symulacji projektów w metodyce Kanban na platformie Jira poprzez wypełnienie ich przykładowymi danymi. Aplikacja zakłada umożliwienie użytkownikowi wprowadzenia parametrów do generowanych danych,
takich jak okres trwania projektu, dostępne statusy oraz liczba zadań do wygenerowania.
