\section*{Wprowadzenie}
Analiza metryk jest ważnym aspektem prowadzenia projektów informatycznych. Pozwalają one na wgląd do wielu parametrów projektu i na tej podstawie podejmowanie ważnych decyzji związanych z jego prowadzeniem.
Nauka tej analizy nie jest prostym zadaniem -- dane są cenniejsze od złota i dostęp do danych realistycznych projektów na potrzeby szkoleń jest bardzo ograniczony. Ponadto, nie istnieją rozwiązania pozwalające na przeprowadzanie
symulacji projektów do celów dydaktycznych na najpopularniejszych platformach.
Praca "Generowanie danych na potrzeby nauki metryk w inżynierii oprogramowania" stara się znaleźć rozwiązanie tego problemu, poprzez opracowanie narzędzia umożliwiającego symulacje projektów informatycznych na platformach
będących w powszechnym użytku komercyjnym. Nauka analizy metryk jest również elementem zajęć na uczelni, a więc istnieje środowisko w którym można szybko przetestować narzędzie w praktyce i trwale udoskonalać.\\
Obecnie, aby przeprowadzić szkolenie z zakresu analizy metryk w inżynierii oprogramowania z wykorzystaniem jednej z najszerzej stosowanych platform, konieczne jest ręczne spreparowanie projektu.
Jest to możliwe do osiągnięcia, jednak ilość danych jaką należy wprowadzić w celu umożliwienia miarodajnej analizy sprawia, że proces ten robi się ekstremalnie czasochłonny, żmudny i podatny na błędy.
Automatyzacja procesu wypełniania projektów relatywnie realistycznymi parametrami pozwoliłaby upowszechnić dostęp do wiedzy o analizie metryk w inżynierii oprogramowania.
Analiza metryk jest o tyle istotna, że pozwala przeprowadzać uwielbianą przez biznes ocenę produktywności w obiektywny sposób. Praca intelektualna nie jest prosta do zmierzenia, stąd niezbędny jest odpowiedni poziom ekspertyzy
aby w ogóle poprawnie dobrać metryki, nie mówiąc już o wyciąganiu z nich rozsądnych wniosków.
"Generowanie danych na potrzeby nauki metryk w inżynierii oprogramowania" kładzie nacisk na popularyzację takiej wiedzy właśnie w tym celu.
