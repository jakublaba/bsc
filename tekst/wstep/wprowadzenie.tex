\section{Wprowadzenie}
Na wydziale w ramach zajęć na studiach II stopnia prowadzone są laboratoria z zakresu metryk w metodyce Kanban.
Dla zwiększenia atrakcyjności i wartości dydaktycznej ćwiczeń, dobrym pomysłem jest przeprowadzenie ich z użyciem jednego z najpopularniejszych narzędzi,
często wykorzystywanych w środowisku komercyjnym (Jira, Azure DevOps, YouTrack).
Ćwiczenie polega na dostarczeniu każdemu studentowi spreparowanego projektu w narzędziu, tak aby wyglądał jak prawdziwy trwający projekt informatyczny,
a następnie analiza poszczególnych metryk oraz ich interpretacja.\\
Spreparowanie takiego projektu jest możliwe manualnie, aczkolwiek jest to proces żmudny, czasochłonny, i podatny na błędy.\\
Jednocześnie narzędzia udostępniają interfejsy umożliwiające automatyzację tego procesu, która nie dość że zdejmie dużą część pracy z prowadzących zajęcia,
to pozwoli na zapewnienie każdemu studentowi unikalnego zestawu danych, co nada ćwiczeniu bardziej indywidualny charakter i lepiej zaktywizuje do samodzielnego myślenia.\\
