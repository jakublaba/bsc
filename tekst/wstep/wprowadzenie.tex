\section*{Wprowadzenie}
Analiza metryk jest ważnym aspektem prowadzenia projektów informatycznych. Pozwalają one na wgląd do wielu parametrów projektu i na tej podstawie podejmowanie ważnych decyzji związanych z jego prowadzeniem.
Nauka tej analizy nie jest prostym zadaniem -- dostęp do danych realistycznych projektów na potrzeby szkoleń jest bardzo ograniczony, często z uwagi na poufne dane. Ponadto, nie istnieją rozwiązania pozwalające na przeprowadzanie
symulacji projektów do celów dydaktycznych na najpopularniejszych platformach.

Praca "Generowanie danych na potrzeby nauki metryk w inżynierii oprogramowania" podejmuje analizę różnych metryk w inżynierii oprogramowania oraz platform do zarządzania projektami pod kątem możliwości automatyzacji tego procesu,
oraz proponuje rozwiązanie tego problemu, poprzez opracowanie narzędzia umożliwiającego symulacje projektów informatycznych na platformach
będących w powszechnym użytku komercyjnym.

Nauka analizy metryk jest również elementem zajęć na uczelni, a więc istnieje środowisko w którym można szybko przetestować narzędzie w praktyce i trwale udoskonalać.
