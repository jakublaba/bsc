\section{Metryki Statyczne}
Metryki statyczne to najprostszy rodzaj metryk, skupiający się na statycznej analizie
kodu źródłowego.

\subsection*{Liczba linii kodu (LOC)}
Najprostszą metryką statyczną jest liczba linii kodu (LOC).
Oczywiście jest to niezbyt miarodajna metryka -- w obrębie pojedynczego języka można napisać
każdą funkcjonalność na wiele różnych sposobów, między różnymi językami będzie się to wahać jeszcze bardziej.
Co więcej -- może się to wahać nawet pomiędzy różnymi wersjami tego samego języka, np. \cite[\texttt{switch} w Javie 17]{jdk17switch},
zapewniający bardziej zwięzłą składnię niż we wcześniejszych wersjach. \\
Poza samym faktem tego, że to samo może mieć różną objętość pod kątem LOC, nie da się równiez jednoznacznie stwierdzić,
czy mniej linii kodu jest zawsze lepsze niż więcej, czy vice versa.

\subsection*{Metryki Halsteada}
\cite[dupa]{halstead}
