\newpage
\section{Metryki Statyczne}
Pierwszym rodzajem omawianych metryk są metryki statyczne. Spośród omawianych w tej pracy metryk wyróżniają się tym,
że jako jedyne nie skupiają się na mierzeniu parametrów pracy zespołu, tylko kodu.
Część z nich jest w dzisiejszych czasach zintegrowana w narzędziach deweloperskich, takich jak \href{https://www.synopsys.com/software-integrity/static-analysis-tools-sast/coverity.html}{Coverity}
lub \href{https://www.sonarsource.com/products/sonarlint/}{SonarLint}.

\subsection{Liczba linii kodu (LOC)}
Najprostszą metryką statyczną jest liczba linii kodu (ang. Lines of Code -- LOC).
Jest to dobra metryka do orientacyjnej oceny rozmiaru projektu czy danej funkcjonalności, jednak nie nadaje
się do dokładniejszych analiz, ze względu na fundamentalne wady:

\subsubsection*{Istnieje wiele rozwiązań każdego problemu}
Jako przykład rozpatrzmy proste zadanie -- odwrócenie ciągu znaków w języku Java. \\
\lstinputlisting[caption=Początkujący programista zapewne podszedłby do problemu nieco bardziej imperatywnie]{ReverseStringImperative.java}
\lstinputlisting[caption=Bardziej doświadczony programista skorzystałby ze znajomości biblioteki standardowej]{ReverseStringSb.java}
\lstinputlisting[caption=Entuzjasta programowania funkcyjnego mógłby skorzystać ze Stream API]{ReverseStringStream.java}

\subsubsection*{Różne języki programowania mogą zasadniczo różnić się składnią}
Przykładowo, w języku Python nie potrzeba nawet pisać funkcji do odwrócenia
ciągu znaków, wystarczy wykorzystać wbudowany operator indeksowania z odpowiednimi parametrami: \texttt{[::-1]}

\subsection{Złożoność cyklomatyczna}
W 1976 Thomas J. McCabe opracował metrykę złożoności kodu, opartą na grafowej reprezentacji przepływu sterowania programu\cite[]{McCabe1976ACM}.
