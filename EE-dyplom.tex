% !TEX program = xelatex
% !TeX encoding = utf8
% !TeX spellcheck = pl-PL

%%%%%%%%%%%%%%%%%%%%%%%%%%%%%%%%%%%%%%%%%%%%%%%%%%%%%%%%%%%%%%%%%%%%%%%%%%%
% Wybierz rodzaj pracy dyplomowej oraz wydział
% Pick thesis type and faculty
%%%%%%%%%%%%%%%%%%%%%%%%%%%%%%%%%%%%%%%%%%%%%%%%%%%%%%%%%%%%%%%%%%%%%%%%%%%
\documentclass[thesis=inz,faculty=ee]{EE-dyplom} 
% thesis=[inz|mgr|bsc|msc]
%  * inz - praca inżynierska
%  * mgr - praca magisterska
%  * bsc - bachelor thesis
%  * msc - master thesis

% Skróty nazw wydziałów zgodne z domenami internetowymi
% Abbreviations of Faculties according to Internet subdomains
% faculty=[
%	arch,
%	gik,
%	ee,
%	wip
%	]

%%%%%%%%%%%%%%%%%%%%%%%%%%%%%%%%%%%%%%%%%%%%%%%%%%%%%%%%%%%%%%%%%%%%%%%%%%%
% Konfiguracja - do personalizacji
% Configuration - to be personalized
%%%%%%%%%%%%%%%%%%%%%%%%%%%%%%%%%%%%%%%%%%%%%%%%%%%%%%%%%%%%%%%%%%%%%%%%%%%
\instytut{Instytut Sterowania i Elektroniki Przemysłowej}
\kierunek{Informatyka Stosowana}
\specjalnosc{Inżynieria Oprogramowania}
\title{Generowanie danych na potrzeby nauki metryk w inżynierii oprogramowania}
\engtitle{Telechronic Optimization of~Universal~History Using~Hyperputer}
\album{311338}
\author{Jakub Łaba}
\promotor{mgr. inż. Krzysztof Marek}
\date{2023}
% TODO uzupełnić dokładną datę
\longdate{2077-07-27}

\grantlicense{FALSE} % [TRUE|FALSE]

%%%%%%%%%%%%%%%%%%%%%%%%%%%%%%%%%%%%%%%%%%%%%%%%%%%%%%%%%%%%%%%%%%%%%%%%%%%
% Streszczenie pracy i abstract.
% In case of thesis in English swap the order - English version goes first.
%%%%%%%%%%%%%%%%%%%%%%%%%%%%%%%%%%%%%%%%%%%%%%%%%%%%%%%%%%%%%%%%%%%%%%%%%%%
\streszczeniepracy{
To jest streszczenie. To jest trochę za krótkie, jako że powinno zająć całą stronę.

\lipsum[1-4]
} % koniec streszczenia

\slowakluczowe{Kanban, Jira, Rust}

\thesisabstract{
This is abstract. This one is a little too short as it should occupy the whole page.

\lipsum[1-4]
} % end of abstract

\thesiskeywords{X, Y, Z}

%%%%%%%%%%%%%%%%%%%%%%%%%%%%%%%%%%%%%%%%%%%%%%%%%%%%%%%%%%%%%%%%%%%%%%%%%%%
% Tu zaczyna się dokument
% Here is the beginning of the document
%%%%%%%%%%%%%%%%%%%%%%%%%%%%%%%%%%%%%%%%%%%%%%%%%%%%%%%%%%%%%%%%%%%%%%%%%%%
\begin{document}
    % Strony nagłówkowe
    % Headers
    \frontpages

    % Właściwa treść jest w pliku tekst/main.tex
    % Real contents is in tekst/main.tex
    \chapter{Wstęp}
\section*{Wprowadzenie}
Analiza metryk jest ważnym aspektem prowadzenia projektów informatycznych. Pozwalają one na wgląd do wielu parametrów projektu i na tej podstawie podejmowanie ważnych decyzji związanych z jego prowadzeniem.
Nauka tej analizy nie jest prostym zadaniem -- dostęp do danych realistycznych projektów na potrzeby szkoleń jest bardzo ograniczony, często z uwagi na poufne dane. Ponadto, nie istnieją rozwiązania pozwalające na przeprowadzanie
symulacji projektów do celów dydaktycznych na najpopularniejszych platformach.

Praca "Generowanie danych na potrzeby nauki metryk w inżynierii oprogramowania" podejmuje analizę różnych metryk w inżynierii oprogramowania oraz platform do zarządzania projektami pod kątem możliwości automatyzacji tego procesu,
oraz proponuje rozwiązanie tego problemu, poprzez opracowanie narzędzia umożliwiającego symulacje projektów informatycznych na platformach
będących w powszechnym użytku komercyjnym.

Nauka analizy metryk jest również elementem zajęć na uczelni, a więc istnieje środowisko w którym można szybko przetestować narzędzie w praktyce i trwale udoskonalać.

\section{Cel pracy}
Celem pracy jest opracowanie automatyzacji wspomnianego we wstępie pracy procesu przygotowywania materiałów do zajęć.
W ramach projektu powstała aplikacja umożliwiająca generowanie projektów w metodyce Kanban na wybranej platformie.
Program umożliwia zdefiniowanie parametrów:
\begin{itemize}
    \item Nazwa projektu
    \item Autor projektu
    \item Data rozpoczęcia projektu
    \item Data zakończenia projektu
    \item Ilość zadań w projekcie
\end{itemize}

\section*{Plan Pracy}
W pierwszym rodziale pracy ma miejsce przegląd różnych metryk w inżynierii oprogramowania, pod kątem ich przydatności w pomiarze pracy zespołu.
Porównywane są tutaj metryki statyczne oraz metryki z metodyk zwinnych Scrum i Kanban.

Drugi rozdział opisuje przegląd platform służących do zarządzania projektami, dokładniejszą analizę funkcjonalności wybranych z nich.
Na końcu rozdziału dokonany i uzasadniony jest wybór platformy do realizacji rozwiązania.

W trzecim rozdziale opisana jest implementacja oraz weryfikacja rozwiązania.

Ostatni rozdział podsumowuje pracę -- jakie cele udało się osiągnąć, jakich udoskonaleń względem zakresu pracy nie udało się zaimplementować, oraz
omawia możliwości dalszego rozwoju projektu.


\chapter{Metryki w Inżynierii Oprogramowania}
\newpage
\section{Metryki Statyczne}
Pierwszym rodzajem omawianych metryk są metryki statyczne. Spośród omawianych w tej pracy metryk wyróżniają się tym,
że jako jedyne nie skupiają się na mierzeniu parametrów pracy zespołu, tylko kodu.
Część z nich jest w dzisiejszych czasach zintegrowana w narzędziach deweloperskich, takich jak \href{https://www.synopsys.com/software-integrity/static-analysis-tools-sast/coverity.html}{Coverity}
lub \href{https://www.sonarsource.com/products/sonarlint/}{SonarLint}.

\subsection{Liczba linii kodu (LOC)}
Najprostszą metryką statyczną jest liczba linii kodu (ang. Lines of Code -- LOC).
Jest to dobra metryka do orientacyjnej oceny rozmiaru projektu czy danej funkcjonalności, jednak nie nadaje
się do dokładniejszych analiz, ze względu na fundamentalne wady:

\subsubsection*{Istnieje wiele rozwiązań każdego problemu}
Jako przykład rozpatrzmy proste zadanie -- odwrócenie ciągu znaków w języku Java. \\
\lstinputlisting[caption=Początkujący programista zapewne podszedłby do problemu nieco bardziej imperatywnie]{ReverseStringImperative.java}
\lstinputlisting[caption=Bardziej doświadczony programista skorzystałby ze znajomości biblioteki standardowej]{ReverseStringSb.java}
\lstinputlisting[caption=Entuzjasta programowania funkcyjnego mógłby skorzystać ze Stream API]{ReverseStringStream.java}

\subsubsection*{Różne języki programowania mogą zasadniczo różnić się składnią}
Przykładowo, w języku Python nie potrzeba nawet pisać funkcji do odwrócenia
ciągu znaków, wystarczy wykorzystać wbudowany operator indeksowania z odpowiednimi parametrami: \texttt{[::-1]}

\subsection{Złożoność cyklomatyczna}
W 1976 Thomas J. McCabe opracował metrykę złożoności kodu, opartą na grafowej reprezentacji przepływu sterowania programu\cite[]{McCabe1976ACM}.

\section{Inne stare metryki}

\subsection{Metryki Halsteada}
W 1977 prof Maurice Howard Halstead z Purdue University, sformułował metryki mające opisywać kod w dokładny sposób\cite[]{halstead1977elements}.
Metryki prof Halsteada opierają się o 4 wartości:
\begin{itemize}
    \item $\eta_1$ -- liczba unikalnych operatorów
    \item $\eta_2$ -- liczba unikalnych operandów
    \item $N_1$ -- całkowita liczba operatorów
    \item $N_2$ -- całkowita liczba operandów
\end{itemize}
Na ich podstawie można obliczyć następujące parametry:
\begin{itemize}
    \item Słownictwo: $\eta = \eta_1 + \eta_2$
    \item Długość: $N = N_1 + N_2$
    \item Szacunkowa długość: $\hat{N} = \eta_1log_2\eta_1 + \eta_2log_2\eta_2$
    \item Objętość: $V = N \times log_2\eta$
    \item Trudność: $D = \frac{\eta_1}{2} \times \frac{N_2}{\eta_2}$
    \item Wysiłek: $E = D \times V$
    \item Czas: $T = \frac{E}{18}[s]$
    \item Bugi: $B = \frac{E^{\frac{2}{3}}}{3000}$ później uproszczone do $B = \frac{V}{3000}$
\end{itemize}
Metryki te nie przyjęły się, ze względu na ich właściwości:
\begin{enumerate}
    \item Objętość programu jest liniowo zależna od długości
    \item Wysiłek jest liniowo zależny od objętości
    \item Czas jest liniowo zależny od wysiłku
\end{enumerate}
Łańcuch tych zależności sprowadza się do tego, że każda z tych metryk jest u podstaw liniowo zależna od długości, która pod kątem przydatności niewiele różni się od liczby linii kodu. \\
Zebranie odpowiednich parametrów kodu i obliczenie tych metryk jest zatem inwestycją czasu, która w porównaniu z LOC jest nieopłacalna.

\section{Metryki w metodyce Scrum}
Chociaż oficjalny Scrum Guide \cite[]{Schwaber2012TheSG} nie definiuje bezpośrednio metryk, powszechnie przyjęło się korzystanie z \textit{Velocity} oraz wykresów wypalenia.

\subsection{Czym są Story Points?}
Mimo że Scrum jest nieco starszy niż Extreme Programming, koncepcja Story Points została zaczerpnięta właśnie z XP.
Na swoim blogu \cite[]{StoryPointsRevisited} jeden z autorów tej metodyki opisuje że początkowo estymaty ustalało się po prostu w liczbie dni.
Nie była to jednak najbardziej precyzyjna jednostka do estymacji, a więc szybko został zaadaptowany koncept tzw. \textit{Dni Idealnych}.
Ron Jeffries opisuje dni idealne następującymi słowami: \textit{"We quickly went to what we called “Ideal Days”, which was informally described as how long it would take a pair to do it if the bastards would just leave you alone."}.
Kolokwialnie mówiąc, dni idealne to dni w których programiści mieliby święty spokój.
Aby z liczby dni idealnych uzyskać estymatę faktycznego czasu, były one mnożone przez współczynnik obciążenia.
Nazwa "dni idealne" często była skracana do "dni", co prowadziło do nieporozumień z klientami, stąd zaczęto nazywać je po prostu "punktami". \\
Obecnie Story Points funkcjonują jako relatywna jednostka, nie mająca bezpośredniego przełożenia na czas wykonania, a używana jedynie do oceny ilości pracy.

\subsection{Velocity}
Velocity jest bardzo prostą metryką, polegającą jedynie na zsumowaniu wagi zadań w danym sprincie, dając wgląd w wydajność zespołu.
Metryka ta niestety jest bardzo podatna na niepoprawne estymowanie zadań -- świadomie bądź nie, co może prowadzić do fałszywej oceny postępów w przypadku
zespołów które jeszcze się nie "skalibrowały" i niedoestymowują zadań, oraz zespołów które celowo je przeestymowują aby wypaść lepiej na papierze.

\subsection{Wykres wypalenia}
Błędnie nazywany metryką, ponieważ jest to jedynie wizualizacja postępu projektu.
Oś Y reprezentuje skumulowaną wagę pozostałych zadań, natomiast oś X czas.
Na wykresie wypalenia można zidentyfikować czy zespół nadąża z wykonywaniem zadań.
W dobrze przebiegającym projekcie powinien być obserwowalny trend spadkowy, moment kiedy wartość na osi Y spadnie do 0 jest momentem wykonania wszystkich zadań na zadany okres.

\subsection{Inne Metryki}
Jako że metryki nie są oficjalnie zdefiniowane, zależą one tak naprawdę od organizacji.
Firmy zajmujące się wspieraniem produkcji oprogramowania (np. \href{https://www.atlassian.com/}{Atlassian}, \href{https://www.sealights.io/}{Sealights}), wymieniają tutaj m.in. \cite[]{ScrumMetricsSealights} \cite[]{ScrumMetricsAtlassian}:
\begin{itemize}
    \item Sprint Goal Success
    \item Defect Density
    \item Time to Market
    \item ROI (Return of Investment)
\end{itemize}
Z powodu tego, że te metryki nie są mierzalne z samych parametrów zadań na platformie, nie są one istotne dla tej pracy.

\section{Metryki w metodyce Kanban}
Kanban jest metodyką zwinną służącą do zarządzania pracą intelektualną, gdzie jednym z podstawowych założeń jest zapewnienie mechanizmów
do wizualizacji tzw. \textit{niewidzialnej pracy}, czyli pracy intelektualnej. Kanban wychodzi z założenia, że pojedyncza metryka nie jest wystarczająca --- każdy
pojedynczy wyznacznik można łatwo oszukać. Z tego powodu, Kanban ma wiele metryk zasilających jego mechanizmy wizualizacji.\\
Projekt koncentruje się na kilku wybranych metrykach, opisanych w kolejnych podrozdziałach.
\begin{itemize}
    \item Wiek Pracy w Toku (Age of WIP)
    \item Przepustowość (Throughput)
    \item Efektywność przepływu (Flow efficiency) --- Stosunek ilości dni kiedy zadanie było w stanie aktywnym do ilości dni w których było w stanie oczekiwania.
    \item Czas realizacji (Lead time) --- Ile dni minęło między przejściem zadania przez punkt zobowiązania, a punkt dostarczenia % TODO tutaj jakieś przypisy czym są te punkty
\end{itemize}

\subsection{Wiek Pracy w Toku (Age of WIP)}
Metryka ta opisuje od jakiego czasu trwa praca nad poszczególnymi zadaniami. Jako że Kanban zakłada wspólną pracę nad zadaniami, metryka ta pozwala szybko 
zidentyfikować które zadania zaczynają się blokować i od razu przydzielać pomoc.

\subsection{Przepustowość (Throughput)}
Metryka ta opisuje ile jednostek wartości dostarczamy w danym czasie. Metryka podobna do \textit{Velocity} w SCRUMie. Jednostką wartości mogą być storypointy, zadania, funkcjonalności, itd.
Metryka ta daje bardzo uproszczony i ogólny wgląd w wydajność zespołu.

\subsection{Efektywność przepływu (Flow efficiency)}
Metryka ta jest nieco bardziej złożona: aby policzyć efektywność przepływu, rozpatrujemy ile czasu zadanie znajdowało się w stanie aktywnym (faktycznie była przeprowadzana nad nim praca), a ile czasu
w stanie oczekiwania (statusy typu: \textit{``Selected for development''}, \textit{``Ready for testing''}, itd.). Wartość tej metryki to stosunek czasu spędzonego w stanie aktywnym do czasu w stanie oczekiwania.
Metryka ta pozwala na identyfikowanie, czy przepustowość systemu wynika z faktycznego osiągnięcia limitu wydajności przez pracowników, czy nieoptymalne procedury przekazywania zadań między etapami.

\subsection{Czas realizacji (Lead time)}
Aby opisać tę metrykę, wpierw konieczne jest zdefiniowanie takich pojęć jak \textit{punkt zobowiązania} oraz \textit{punkt dostarczenia}.
Kanban jest zorientowany biznesowo, stąd punkt zobowiązania nie jest momentem faktycznego rozpoczęcia pracy nad zadaniem, tylko momentem zadeklarowania się przez zespół, że zadanie zostanie wykonane.
Przykładowo, w SCRUMie punktem zobowiązania byłby sprint planning, a nie moment przeciągnięcia zadania ze statusu \textit{``TODO''} do \textit{``In Progress''}.\\
Punkt dostarczenia, to punkt oddania skończonego zadania --- w zależności od specyfiki zespołu może to oznaczać różne rzeczy, np.\ jeżeli zespół kontroluje cały produkt, to punktem dostarczenia
będzie faktyczne oddanie funkcjonalności klientowi, natomiast jeżeli mamy do czynienia z zespołem zajmującym się jedynie programowaniem, to punktem dostarczenia będzie przekazanie funkcjonalności do testerów.\\
Lead time to czas jaki zadanie spędziło pomiędzy tymi dwoma punktami --- od zobowiązania do dostarczenia.
Ta metryka daje bardziej wartościowy wgląd w efektywność zespołu od przepustowości, patrząc z biznesowego punktu widzenia.

\section{Wybór Metryk}



    % Bibliografia - musi być
    % Bibliography - must exist
    \bibliografia

    % Strony końcowe - można zakomentować, jeśli zbędne
    % Additional pages - comment out if not needed
    
    % Wykaz symboli i skrótów - patrz opis w tekście przykładowym
    %\acronymslist
    % Spis rysunków
    %\listoffigures
    % Spis tabel
    %\listoftables
    % Załączniki (plik appendices.tex)
    %\easyappendices
\end{document}
%%%%%%%%%%%%%%%%%%%%%%%%%%%%%%%%%%%%%%%%%%%%%%%%%%%%%%%%%%%%%%%%%%%%%%%%%%%
